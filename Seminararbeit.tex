\documentclass[a4paper,oneside,numbers=noenddot,fontsize=12pt,open=right]{scrreprt}
\usepackage{ngerman}
\usepackage[utf8]{inputenc}
\usepackage[T1]{fontenc}

% do not use bitmap font, use vector font
\usepackage{lmodern}

% für ordentliche tabellen
\usepackage{tabularx}

\usepackage{multirow}

\usepackage[hyperfootnotes=false]{hyperref}

% für korrekten zeilenabstand
\usepackage{setspace}

% festlegung bibtex style
\bibliographystyle{alpha}

% fortlaufende nummerierung aller abb und fußnoten
\usepackage{chngcntr}
\counterwithout{footnote}{chapter}
\counterwithout{figure}{chapter}

% seitenränder
\usepackage{geometry}
\geometry{a4paper,left=30mm,right=20mm, top=25mm, bottom=25mm}

% um grafiken zu includieren
\usepackage{graphicx}

% für checkboxen
\usepackage{amssymb}

% für graphen
%\usepackage{tikz}
%\usetikzlibrary{arrows}
%sequenzdiagramm
%\usepackage{pgf-umlsd} 

% einbettung von quelltext
\usepackage{listings}
\usepackage{colortbl}
\usepackage{xcolor}

% für tabellen im querformat
\usepackage{pdflscape}

\usepackage{pgfplotstable}
\usepackage{pgfplots}

% für hübsche ja und nein symbole
\usepackage{pifont}
\newcommand{\cmark}{\textcolor{green}{\ding{51}}}
\newcommand{\xmark}{\textcolor{red}{\ding{55}}}

% custom fontsizes
\usepackage{anyfontsize}

% seitenhintergrundbilder
% für diesen bescheuerten sperrvermerk
%\usepackage{wallpaper}
%\usepackage[firstpage]{draftwatermark}

%\SetWatermarkText{~SPERRVERMERK}
%\SetWatermarkScale{0.6}
%\SetWatermarkAngle{50}

% für anforderungsanalyse
\newcounter{anforderung}
\newcommand{\anf}[1]
{
    \refstepcounter{anforderung}
%    (A\arabic{anforderung})
    \subsubsection*{UC--\arabic{anforderung} #1}
    \label{UC:\arabic{anforderung}}
}

\newcommand{\state}[3]
{
\subsubsection*{#1}
%~\\
%
\begin{tabularx}{\textwidth}{|l|X|}
\hline 
\textbf{Beschreibung:} & {#2} \\
\hline 
\textbf{Eigenschaften:} & {#3} \\
\hline
\end{tabularx}
\\\\
}

\newcommand{\anfanf}[4]
{
\begin{tabularx}{\textwidth}{|l|X|}
\hline 
\textbf{Frequenz:} & {#1} \\
\hline 
%\textbf{Akteur:} &    \\
\textbf{Beschreibung:} & {#2} \\
\hline 
\textbf{Bedingung:} & {#3} \\
\hline 
\textbf{Ergebnis:} & {#4} \\
\hline
\end{tabularx}
\\\\
}

\newcommand{\refanf}[1]
{UC--\ref{#1}}

% nomain, if you define glossaries in a file, and you use \include{INP-00-glossary}
%\usepackage[
%nonumberlist, %keine Seitenzahlen anzeigen
%acronym,      %ein Abkürzungsverzeichnis erstellen
%nopostdot,
%toc]          %Einträge im Inhaltsverzeichnis
%%section      %im Inhaltsverzeichnis auf section-Ebene erscheinen
%{glossaries}

%\makeglossaries
%%Befehle für Glossar
%\newglossaryentry{glos:fab}{
%name=Fabricate,
%description={Beschreibt alles was außerhalb des AIMS\texttrademark \ EUV stattfindet.}
%}
%
%\newglossaryentry{glos:toolpc}{name=Tool-PC, description={Jener Computer, auf dem die AIMS\texttrademark \ EUV Nutzerapplikation läuft.}}
%
%\newglossaryentry{glos:outerhh}{name=OuterHandlingHost, description={Ein Computer, auf dem ein Windows Embedded läuft, der Befehle für die Ansteuerung des OuterHandlings aus dem Netzwerk entgegen nimmt und an die entsprechenden Subsysteme weiter gibt und Statusänderungen des Outer-Handlings den angemeldeten Clients bekannt macht.}}
%
%\newglossaryentry{glos:aimshost}{name=AimsHost, description={Ist ein Dienst, welcher als High-Level-Schnittstelle für die AIMS\texttrademark \ EUV Nutzerapplikation dient, um Maschinenspezifische Befehle weiterzugeben.}}
%
%\newglossaryentry{glos:ll}{name=Load Lock, description={dt.: Ladeschleuse; bezeichnet jene Schleuse, die die Verbindung zwischen Inner-Handling und Outer-Handling herstellt.}}
%
%\newglossaryentry{glos:ad}{name=Anwendungsdomäne, description={Ist eine isolierte Laufzeitumgebung für .NET Anwendungen, um Sicherheit und Zuverlässigkeit für diese Anwendungen zu bieten.}}
%
%\newglossaryentry{glos:framework}{name=Framework, description={Beschreibt ein Programmiergerüst, welches in der Softwaretechnik, insbesondere in der objektorientierten Softwareentwicklung Anwendung findet.}}
%
%\newglossaryentry{glos:remote}{name=Remote, description={Beschreibt alles, was sich außerhalb der Anwendungsdomäne befindet.}}
%
%\newglossaryentry{glos:hap}{name=HAP GmbH Dresden, description={Unternehmen aus Dresden, welches die Ansteuerung des Outer-Handlings mitentwickelt hat.}}
%
%\newglossaryentry{glos:clr}{name=Common Language Runtime, description={Die Laufzeitumgebung von .NET für die Sprachen VisualBasic.NET, C\# und J\#.}}
%
%\newglossaryentry{glos:or}{name=oder, description={Eine Disjuktion (logisches d.h. einschließendes oder). Nicht zu verwechseln mit der Kontravalenz (ausschließendes oder).}}
%
%\newglossaryentry{glos:uri}{name=Uniform Resource Identifier, description={Ist eine Zeichenfolge, die zur eindeutigen Identifizierung einer abstrakten oder physischen Ressource dient. Bsp: https://www.github.com/rails/rails.git}}

%Eine Abkürzung mit Glossareintrag
\newacronym{soa}{SoA}{Struct of Array}
\newacronym{aos}{AoS}{Array of Struct}



\definecolor{mygreen}{rgb}{0,0.6,0}

% damit der quellcode nicht mit "Listing: x" gekennzeichnet wird
\renewcommand{\lstlistingname}{Codefragment}
\lstset {
    backgroundcolor=\color{white},   % choose the background color; you must add \usepackage{color} or \usepackage{xcolor}
  basicstyle=\footnotesize,        % the size of the fonts that are used for the code
  breakatwhitespace=false,         % sets if automatic breaks should only happen at whitespace
  breaklines=true,                 % sets automatic line breaking
  captionpos=b,                    % sets the caption-position to bottom
  commentstyle=\color{mygreen},    % comment style
  deletekeywords={...},            % if you want to delete keywords from the given language
  escapeinside={\%*}{*)},          % if you want to add LaTeX within your code
  extendedchars=true,              % lets you use non-ASCII characters; for 8-bits encodings only, does not work with UTF-8
  frame=single,                    % adds a frame around the code
  identifierstyle=\color{black},
  keepspaces=true,                 % keeps spaces in text, useful for keeping indentation of code (possibly needs columns=flexible)
  keywordstyle=\color{blue},       % keyword style
%  language=[GNU]C++,                 % the language of the code
  language=[Sharp]C,                 % the language of the code
  morekeywords={*,...},            % if you want to add more keywords to the set
  numbers=left,                    % where to put the line-numbers; possible values are (none, left, right)
  numbersep=5pt,                   % how far the line-numbers are from the code
  numberstyle=\tiny\color{gray}, % the style that is used for the line-numbers
  numberbychapter = false,         % false: fortlaufende nummerierung aller listings
  rulecolor=\color{black},         % if not set, the frame-color may be changed on line-breaks within not-black text (e.g. comments (green here))
  showspaces=false,                % show spaces everywhere adding particular underscores; it overrides 'showstringspaces'
  showstringspaces=false,          % underline spaces within strings only
  showtabs=false,                  % show tabs within strings adding particular underscores
  stepnumber=2,                    % the step between two line-numbers. If it's 1, each line will be numbered
  stringstyle=\color{orange},     % string literal style
  tabsize=3,                       % sets default tabsize to 2 spaces
  title=\lstname                   % show the filename of files included with \lstinputlisting; also try caption instead of title
}

\setcounter{tocdepth}{3}
\setcounter{secnumdepth}{3}

\title{\thema}

\def \author{Tom Möbert}
\def \matrikelnummer{167510}
\def \betrieb{Carl Zeiss SMT GmbH}
\def \hochschule{Friedrich Schiller Universität Jena}
\def \hochschuleGenitivS{\hochschule}
\def \fmi{Fakultät für Mathematik und Informatik}
\def \studienbereich{Technik}
\def \studiengang{Informatik}
\def \arbeit{Seminararbeit}
\def \apfelsaftpresse{\gls{sda}}
\def \apfel{\acrshort{sda}}

\def \aimseuv{AIMS\texttrademark~EUV}
\def \thema{Grafikkartenprogrammierung mit Cuda}
\def \prove{PROVE\textsuperscript{\textregistered}}
\begin{document}
\begin{titlepage}
%\ThisLRCornerWallPaper{1}{img/Sperrvermerk.pdf}
    \begin{center}
    ~\vspace{1cm}\\
   {\fontsize{35}{45}\selectfont \textbf{\MakeUppercase \arbeit}}\\
    ~\vspace{0.5cm}\\
    zum Thema\\
    ~\vspace{0.5cm}\\
%    \begin{LARGE}
    \textbf{\glqq\thema\grqq}\\
%    \end{LARGE}  \\
    ~\vspace{0.5cm}\\
    vorgelegt an der\\
    {\fmi}\\
    {der \hochschuleGenitivS}
    \vspace{2cm}

 \normalsize{
    \begin{tabular}{lrl}
    	\textbf{von:} &~& {\author} \\
    	&~& A.-S.-Makarenko-Str. 63 \\
    	&~& 07546 Gera \\ \\ \\
%    	\textbf{am:} &~& {\today} \\ \\
    	\textbf{Matrikelnr.:}&~& {\matrikelnummer} \\ \\
    	\textbf{Studiengang:} &~& {\studiengang} \\ \\ \\ \\
    	\textbf{Praxispartner:} &~& Carl Zeiss Jena GmbH \\
    	&~& {\betrieb} \\
    	&~& {Carl-Zeiss-Promenade 10} \\
		&~& {07745 Jena} \\ \\ \\
		\textbf{Gutachter der }\\ \textbf{\hochschule:} &~& Herr Prof. Dr.-Ing. Martin Bücker \\ \\ \\ \\
    	\textbf{Fachlicher Betreuer:}\\ &~& Herr Dipl.-Inf. Frank Taubert \\ \\
    \end{tabular}\\
    }
  \end{center}
\end{titlepage}

\tableofcontents

%Zeilenabstand
%\setstretch{1.5}%
\onehalfspacing



\chapter{Sinn und Zweck paralleler Programmierung}
\label{intro}
Seit Beginn der Rechentechnik Anfang der 40er Jahre verfolgen Informatiker das Ziel die zur Verfügung stehende Rechenleistung effizient zu nutzen.

Die parallele Programmierung stellt eine wesentliche Maßnahme dar, um dieses Ziel zu erreichen. Mit dieser Arbeit  soll die Umsetzung eines gegebenen Problems auf einer Grafikkarte mittels nVidia CUDA vorgestellt werden.

Parallele Programme beschreiben potentiell gleichzeitig ablaufende Aktivitäten, die miteinander kooperieren, um eine gemeinsame Aufgabe zu lösen. Dies setzt Programme voraus, die unabhängig von der Anzahl und der Geschwindigkeit der Prozessoren die gewünschten Ergebnisse liefern. Die Softwareentwicklung für parallele Algorithmen ist deshalb im Vergleich zur sequentiellen Programmierung wesentlich komplexer.

Durch die Aufteilung eines Programms in sogenannte \textbf{Threads} (\glqq leichtgewichtige Prozesse\grqq) lassen sich die Ressourcen heutiger Mirkoprozessoren effizient nutzen.
Threads besitzen dabei folgende Eigenschaften:
\begin{itemize}
\item sind sequentielle Befehlsausführungen
\item stellen Einheit für die Prozessorzuteilung dar
\item laufen in einem Prozessadressraum ab
\end{itemize}
Mit der Einführung von Threads werden im wesentlichen zwei Ziele verfolgt:
\begin{enumerate}
\item Strukturierung unabhängiger Programme und Programmkomponenten
\item Leistungssteigerung durch effiziente Parallelarbeit
\end{enumerate}

%Besonders aus der Möglichkeit der effizienten Parallelarbeit geht ein weiterer wesentlicher Vorteil hervor:
Die Leistungsfähigkeit von Programmen ist hauptsächlich abhängig von der Taktrate der CPU. Erhöht sich die Taktrate, wird ebenso das Programm schneller.
Allerdings haben die Taktraten heutiger CPUs einen Grenzwert erreicht.
Größere Taktraten würden diverse technische Probleme mit sich bringen, wie beispielsweise erhöhte Wärmeentwicklung auf dem Prozessor und damit verbundene, eine geeignete Wärmeabfuhr zu finden.
Daher gehen Chiphersteller schon seit einigen Jahren den Weg, möglichst viele Recheneinheiten (Kerne) auf einer CPU unterzubringen. Grafikprozessoren sind von Haus aus mit einer deutlich größeren Anzahl an Kernen ausgestattet.
Programme die entsprechend implementiert wurde diese Technologien effizient zu nutzen, können einen deutlichen Geschwindigkeitsvorteil gegenüber einer parallelen Ausführung auf einer CPU aufweisen.


\chapter{Aufgabenstellung}
Gegeben Kraft-Abstandskurven eines Rasterkraftmikroskops. Rasterkraftmikroskope werden zur Untersuchung von Oberflächen genutzt, um bspw. ein Höhenprofil erstellen zu können. Hierzu wird die vorliegende Oberfläche an möglichst vielen Stellen mit einer ca. 40 nm großen Spitze abgetastet, d.h. es wird zu jeder Messhöhe die der Abtastspitze entgegengebrachte Kraft gemessen.
Der Verlauf einer solchen Kraft-Abstandskurve lässt sich in drei lineare Funktionen unterteilen und ist qualitativ in Abbildung \ref{fig:kraftqual} dargestellt.

\begin{figure}[h!]
\centering
\includegraphics[scale=0.5]{qual.pdf}
\caption{Qualitativer Verlauf einer Kraft-Abstandskurve}
\label{fig:kraftqual}
\end{figure}

Der grüne Kurvenverlauf stellt die Annäherung der Abtastspitze zur Probe dar. Auffällig ist, dass die entgegengebrachte Kraft bei diesem Vorgang konstant bleibt.
Die gelbe Kurve zeigt, wie die Spitze von den Adhäsionskraften zwischen ihr und der Probe erfasst wird. Die entgegengebrachte Kraft fällt daher rapide ab, bis die Spitze Kontakt mit der Probe an der Stelle $c$ hergestellt hat.
Der rote Kurvenabschnitt zeigt, wie die auf die Spitze einwirkende Kraft nach Kontaktherstellung stark zunimmt, während weiterhin versucht wird, sich der Probe anzunähern.

Ziel ist es nun, die gegebene Punktewolke eines realen Rasterkraftmikroskops in diese drei linearen Funktionen zu zerlegen.
Ein Ausschnitt einer solchen realen Kraft-Abstandskurve ist in Abbildung \ref{fig:kraftbsp} dargestellt.


\begin{figure}[h!]
\centering
 \begin{tikzpicture}[scale=1]
 \begin{axis}[
   width=15cm,
   xlabel=Messhöhe in m,
   ylabel=Kraft in N]
 \addplot table [y=$Q_A$, x=P]{ex.txt};
 %\addlegendentry{$Q_A$ series}
 \end{axis}
 \end{tikzpicture}
\caption{Ausschnitt einer Kraft-Abstandskurve}
\label{fig:kraftbsp}
\end{figure}

Aufgenommen wurden solche Abtastungen an 256x*256y Positionen, wobei sowohl eine Messung für das Anfahren auf die Probe als auch das Ablassen von ihr aufgezeichnet wurde. Es liegen also insgesamt 131072 Datensätze vor, deren je 300 Abtastwerte mittels dreifach linearer Regression auf einer GPU in die oben beschriebene Form gebracht werden sollen, indem der Kontaktpunkt mit der Probe und der Splitindex gefunden werden.


\section{Einlesen der Datensätze}
Die Datensätze liegen als Textdeateien vor, wobei an jede Messposition in einer eigenen Datei gepsiechert ist. Es liegen somit 65536 Textdateien vor, in denen die Messwerte für das An-/ und Abfahren von der Probe zeilenweise ähnlich wie in einer CSV Datei gepseichert sind.
Das Einlesen dieser vielen 100KiB kleinen Dateien wird durch das häufige öffnen und schließen der Dateien und dem damit verbundenen Overhead seitens des Betreibssystems verlangsamt. Zudem werden für die nötigen Berechnungen nur die ersten zwei der insgesamt 14 Datenspalten benötigt, was Caching seitens des OS zusätzlich erschwert.


Das Einlesen aller 65536 Textdateien benötigt auf einer SSD\footnote{Testsystem: openSUSE 13.2, Linux Kernel 3.16.7 x86\_64, OCZ Vertex 3} 2 Minuten. Dies kann auf 40 Sekunden reduziert werden, wenn das Einlesen mittels openMP Tasks parallelisiert erfolgt.
Der Einlesevorgang auf einer konventionellen HDD benötigt aufgrund der trägen Mechanik hingegen mind. 10 Minuten.

Um dies zu umgehen ist eine Vorverarbeitung nötig, die darin besteht, die Textdateien einmalig einzulesen, die Daten zu filtern und zu sortieren und sie anschließend als Binär Blob auszugeben. Der Einlesevorgang kann somit auf wenige Sekunden reduziert werden.

\section{Implementierung des Kernels}
Als Eingabe erhält der Kernel die aus dem Binär Blob gelesenen Datensätze.
Der Kernel hat die Aufgabe die Eingangs beschriebene dreifache lineare Regression für jeden Datensatz durchzuführen.

\begin{lstlisting}[caption=Implementierung des Kernel in Pseudocode,label=kernel]
void kernel(const point_t* pts, const int nSets)
{
    int myAddr = threadIdx.x+blockIdx.x*blockDim.x;

    if(myAddr < nSets)
    {
        const my_size_t nPoints = pts[myAddr].n;

        contactIdx = calcContactPoint(pts[0:nPoints])
        __syncthreads();

        fitPoints(pts[0:contactIdx]);
        __syncthreads();
                
        my_size_t splitIdx = contactIdx+10;
        fitPoints(pts[contactIdx:splitIdx]);

        fitPoints(pts[splitIdx:nPoints]);
    }
}
\end{lstlisting}


Aufgrund der vergleichsweise geringen Anzahl an Messwerten pro Datensätze (=300) erfolgt eine naive Parallelisierung der Datensätze: Jeder Thread der Grafikkarte bearbeitet einen Datensatz. Wie in Codelisting \ref{kernel} dargestellt wird hierfür

\begin{itemize}
\item die gegebene Punktwolke abgeleitet, um den Kontaktpunkt zu bestimmen (Zeile 7),
\item eine lineare Regression (polyfit) der Kurve vom Ende der Messung bis zum Kontakt mit dem Medium durchgeführt (Zeile 12),
\item der Split-Index erraten\footnote{Die Berechnung des Split-Indexes erwies sich als zu aufwendig und instabil, daher wurde auf eine Implementierung im Rahmen dieses Seminars verzichtet.} (Zeile 15),
\item ein polyfit der Kurve zwischen dem Kontaktpunkt bis Split-Index  (Zeile 16) und
\item ein polyfit der Kurve zwischen Split-Index und Anfang der Messung durchgeführt  (Zeile 18).
\end{itemize}

Grafikkarten arbeiten im SIMD Verfahren. Um zu verhindern, dass die Threads divergieren, da sich die Kontaktpunkte und die Länge der zu interpolierenden Kurve zwangsläufig zwischen den Datensätzen unterscheiden, wurde in den Zeilen 10 und 13 Synchronisationsbarrieren auf Thread-Block Ebene eingeführt.
Diese konnten in der Praxis jedoch keine messbare Veränderung der Laufzeit hervorbringen, da bei nur 300 Messwerten pro Datensatz eine mögliche Divergenz der GPU-Threads zu gering ist.

\section{Datenmodell}
Um eine effiziente Verarbeitung auf der Grafikkarte zu erreichen, müssen die Datensätze in geeigneter Weise strukturiert werden.
Dazu muss entschieden werden, ob sie Daten als \gls{aos} oder als \gls{soa} übergeben werden. Nachfolgend werden diese Ansätze diskutiert.

\subsection{\acrlong{aos}}
\begin{lstlisting}[label=aos_row,caption=\gls{aos} row major]
struct tuple_t { float z,f; };
tuple_t datasets[M][N];
\end{lstlisting}

%N: anzahl an datensätze
%M: anzahl an messwerten pro datensatz

Die Auslegung der Daten als \gls{aos} bedeutet, dass die Daten als zweidimensionales Array übergeben werden, deren Element ein Verbunddatentyp ist, welcher die gemessene Kraft f an der entsprechenden Messhöhe z enthält. Wird das Array wie in Listing \ref{aos_row} dargestellt row major im Speicher abgelegt, Liegen die N Datensätze zeilenweise im Speicher, sodass die i-te Speicherzeile die M Messwerte des i-ten Datensatzes enthält.
Dies ist jedoch für Grafikkarten ungeeignet, da die Messwerte für die einzelnen Threads zu weit auseinander liegen. D.h. wird eine Cacheline geladen, kann hiervon nur das erste Element (der erste Messwert) verwendet werden. Die anderen Threads benötigen ebenfalls den ersten Messwert allerdings den des ihnen zugeordneten Datensatzes. Diese befinden sich in anderen Zeilen des Speichers.
Es müssen somit weitere Cachelines geladen werden, was schlimmsten falls zu einer Ausserialisierung der Threads führt.


\begin{lstlisting}[label=aos_col,caption=\gls{aos} column major]
struct tuple_t { float z,f; };
tuple_t datasets[N][M];
\end{lstlisting}

Es empfiehlt sich daher die Datensätze column major im Speicher abzulegen, wie in Listing \ref{aos_col} dargestellt. Die i-te Zeile des Speichers enthält somit die i-ten Messwerte aller Datensätze. Da jeder Thread einen Datensatz bearbeitet, können alle 32 Threads eines Warps durch eine Cacheline mit Daten versorgt werden.


\subsection{\acrlong{soa}}
Bei NVidia Grafikkarten vor der Pascal Architektur erhält jeder Thread durch einen Lesevorgang ein 32 bit Wort aus der Cacheline\footnote{\href{http://docs.nvidia.com/cuda/pascal-tuning-guide/index.html\#shared-memory-bandwidth}{http://docs.nvidia.com/cuda/pascal-tuning-guide/index.html\#shared-memory-bandwidth}}. tuple\_t hat jedoch eine Größe von 8 Bytes. 
Es braucht also zwei Lesevorgänge um alle Threads eines Warps mit Daten zu versorgen.
Um einen Performancevergleich zu der älteren Kepler Architektur zu erhalten, wird daher auch eine \gls{soa} Variante, wie in Listing \ref{soa} dargestellt, implementiert.
%daher folgende Optimierung:

\begin{lstlisting}[label=soa,caption=\gls{soa}]
struct tuple_t { float z[N],f[N]; };
tuple_t datasets[M];
\end{lstlisting}

Hier ist tuple\_t ein reines \gls{soa}, welches M mal existiert und jeweils Zeiger auf Arrays enthält, welche die N Messwerte z und f des jeweiligen Datensatzes enthalten. Diese können somit von N Threads single strided bearbeitet werden, wie schematisch in Listing \ref{soa_beispiel} dargestellt.
Zwar hat tuple\_t nun eine Größe von 16 Byte, da es zwei 8 Byte Zeiger enthält. Dieses Element muss jedoch nur einmalig gelesen werden, da die enthaltenen Zeiger von den N Threads gemeinsam verwendet werden können.

\begin{lstlisting}[label=soa_beispiel,caption=Bearbeitung eines \gls{soa}]
tuple_t datasets[M];
extern int threadId;

for (i=0; i<M; i++)
{
    float myZ = datasets[i].z[threadId];
    float myForce = datasets[i].f[threadId];
    
    // doWork(myZ, myForce);
}
\end{lstlisting}


\chapter{Benchmarking}
%\section{lesen der textdateien}
%von konventioneller HDD: 8MiB/s
%von SSD:  35 MiB/s -O0
%70 MiB/s -O2
%41 s bei 190 MiB/s -O2 paralleles lesen mit openmp tasks
%
%referenz implementierung 11 MiB/s von SSD
%
%referenz implementierung 11 minuten insgesamt


\section{berechnung}
65536*2 Datensätze:
AoS Ansatz: GPU: 3.5 ms CPU Single: 545 ms CPU 4HWThreads: 199 ms
SOA Ansatz: GPU: 2.27 ms CPU: ??

benchmarking mit double? verdoppelt zeit da wir genug double precision units haben aber eben zwei cachlines lesen müssen anstatt von nur einer

\section{CPU vs. GPU bei variierender Problemgröße}


NVidia GeForce GTX 1060: 1280 HW-Threads; 6GB GDDR5
Intel Core i5-3570K: 3.40 GHz; 4 HW-Threads

\begin{figure}[h!]
\centering
 \begin{tikzpicture}[scale=1]
 \begin{axis}[
   width=15cm,
   legend style={at={(0,1)},anchor=north west},
   /pgf/number format/.cd,
   use comma,
   1000 sep={},
   xlabel=Anzahl Datensätze,xmode = log,log basis x={2},ymode = log,log basis y={2},
   ylabel=Berechnungszeit in ms]
 \addplot table [x=N,y=Kaos]{test_privat};
  \addplot table [x=N,y=Ksoa]{test_privat};
   \addplot table [x=N,y=Caos]{test_privat};
    \addplot table [x=N,y=Csoa]{test_privat};
    \addplot table [x=N,y=cpy]{test_privat};
    \legend{GPU Kernel AoS,GPU Kernel SoA,CPU AoS, CPU SoA, cudaMemcpy}
    
 %\addlegendentry{$Q_A$ series}
 \end{axis}
 \end{tikzpicture}
~\\~\\
 \pgfplotstabletypeset[
columns/N/.style={/pgf/number format/fixed, /pgf/number format/.cd,use comma,1000 sep={}},
columns/Kaos/.style={column name={GPU Kernel AoS},/pgf/number format/.cd,use comma,1000 sep={}},
columns/Ksoa/.style={column name={GPU Kernel SoA},/pgf/number format/.cd,use comma,1000 sep={}},
columns/Caos/.style={column name={CPU AoS},/pgf/number format/.cd,use comma,1000 sep={}},
columns/Csoa/.style={column name={CPU SoA},/pgf/number format/.cd,use comma,1000 sep={}},
columns/cpy/.style={column name={cudaMemcpy},/pgf/number format/.cd,use comma,1000 sep={}}
]{test_privat}
\caption{Benchmark CPU vs. GPU auf Privatrechner}
\label{fig:benchmark:privat}
\end{figure}


Die Berechnungszeiten steigen erwartungsgemäß linear zur Problemgröße an. Die GPU ist jedoch mit 27 ms bei $2^{20}$ Datensätzen deutlich schneller als die CPU mit bestenfalls 2,8 s. Überraschend ist auch, dass der \gls{soa} mit dem \gls{aos} Ansatz auf der GPU nahezu gleich auf liegt. Eine auf der Pascal Architektur erwartete Präferenz für den \gls{aos} Ansatz ist nicht zu erkennen.
Zu beobachten ist auch, dass die \gls{soa} Variante auf der CPU erst ab $2^{18}$ Datensätzen besser performt als \gls{aos}. Diese Beobachtung deckt sich mit dem Benchmark auf gpu03, siehe Abbildung \ref{fig:benchmark:gpu03}.


\begin{figure}[h!]
\centering
 \begin{tikzpicture}[scale=1]
 \begin{axis}[
   width=15cm,
   legend style={at={(0,1)},anchor=north west},
   /pgf/number format/.cd,
   use comma,
   1000 sep={},
   xlabel=Anzahl Datensätze,xmode = log,log basis x={2},ymode = log,log basis y={2},
   ylabel=Berechnungszeit in ms]
 \addplot table [x=N,y=Kaos]{test_gpu03};
  \addplot table [x=N,y=Ksoa]{test_gpu03};
   \addplot table [x=N,y=Caos]{test_gpu03};
    \addplot table [x=N,y=Csoa]{test_gpu03};
    \legend{GPU Kernel AoS,GPU Kernel SoA,CPU AoS, CPU SoA}
    
 %\addlegendentry{$Q_A$ series}
 \end{axis}
 \end{tikzpicture}
 ~\\~\\
  \pgfplotstabletypeset[
columns/N/.style={/pgf/number format/fixed, /pgf/number format/.cd,use comma,1000 sep={}},
columns/Kaos/.style={column name={GPU Kernel AoS},/pgf/number format/.cd,use comma,1000 sep={}},
columns/Ksoa/.style={column name={GPU Kernel SoA},/pgf/number format/.cd,use comma,1000 sep={}},
columns/Caos/.style={column name={CPU AoS},/pgf/number format/.cd,use comma,1000 sep={}},
columns/Csoa/.style={column name={CPU SoA},/pgf/number format/.cd,use comma,1000 sep={}},
columns/cpy/.style={column name={cudaMemcpy},/pgf/number format/.cd,use comma,1000 sep={}}
]{test_gpu03}
\caption{Benchmark CPU vs. GPU auf gpu03.inf-ra.uni-jena.de}
\label{fig:benchmark:gpu03}
\end{figure}


Die beiden Peaks bei $2^{10}$ Datensätzen auf dem Privatrechner und bei $2^{17}$ auf gpu03 lassen sich durch Cache-Assoziativität-Effekte erklären. Normalerweise entscheidet die Speicheradresse eines Datums in welchen Teil des Caches sie abgelegt werden. Bei den erwähnten Peaks scheinen die Arrays unter den CPU Threads so ungünstig aufgeteilt zu sein, dass die Daten der einzelnen Threads aufgrund ähnlicher Speicheradressen immer an die gleiche Stelle im Cache geschrieben werden. Der Cache wird somit künstlich verkleinert und es kommt vermehrt zu Cachmisses.

%Diese Vermutung bestätigt sich durch die Wahl des dynamischen schedules, bei dem eine ungleichmäßige Aufteilung des Arrays unter den Threads erfolgt. Das hierbei entstehende Benchmark ist in Abbildung \ref{fig:benchmark:privat:dynamic} dargestellt.
%
%
%\begin{figure}[h!]
%\centering
% \begin{tikzpicture}[scale=1]
% \begin{axis}[
%   width=15cm,
%   legend style={at={(0,1)},anchor=north west},
%   /pgf/number format/.cd,
%   use comma,
%   1000 sep={},
%   xlabel=Anzahl Datensätze,xmode = log,log basis x={2},ymode = log,log basis y={2},
%   ylabel=Berechnungszeit in ms]
% \addplot table [x=N,y=Kaos]{test_privat_dynamic};
%  \addplot table [x=N,y=Ksoa]{test_privat_dynamic};
%   \addplot table [x=N,y=Caos]{test_privat_dynamic};
%    \addplot table [x=N,y=Csoa]{test_privat_dynamic};
%    \legend{GPU Kernel AoS,GPU Kernel SoA,CPU AoS, CPU SoA}
%    
% %\addlegendentry{$Q_A$ series}
% \end{axis}
% \end{tikzpicture}
%\caption{Benchmark CPU vs. GPU auf Privatrechner (dynamic schedule)}
%\label{fig:benchmark:privat:dynamic}
%\end{figure}
%
%
%\begin{figure}[h!]
%\centering
% \begin{tikzpicture}[scale=1]
% \begin{axis}[
%   width=15cm,
%   legend style={at={(0,1)},anchor=north west},
%   /pgf/number format/.cd,
%   use comma,
%   1000 sep={},
%   xlabel=Anzahl Datensätze,xmode = log,log basis x={2},ymode = log,log basis y={2},
%   ylabel=Berechnungszeit in ms]
% \addplot table [x=N,y=Kaos]{test_gpu03_dynamic};
%  \addplot table [x=N,y=Ksoa]{test_gpu03_dynamic};
%   \addplot table [x=N,y=Caos]{test_gpu03_dynamic};
%    \addplot table [x=N,y=Csoa]{test_gpu03_dynamic};
%    \legend{GPU Kernel AoS,GPU Kernel SoA,CPU AoS, CPU SoA}
%    
% %\addlegendentry{$Q_A$ series}
% \end{axis}
% \end{tikzpicture}
%\caption{Benchmark CPU vs. GPU auf gpu03 (dynamic schedule)}
%\label{fig:benchmark:privat:dynamic}
%\end{figure}


\chapter{Fazit}




% Literaturliste endgueltig anzeigen
\bibliography{lit}
\nocite{*}
% Literaturliste soll im Inhaltsverzeichnis auftauchen
\addcontentsline{toc}{chapter}{Literaturverzeichnis}


\listoffigures
\addcontentsline{toc}{chapter}{Abbildungsverzeichnis}

\lstlistoflistings
\addcontentsline{toc}{chapter}{Listings}

%%glossar ausgeben auch wenn nicht im text referriert
%\glsaddall
%%Glossar ausgeben
%\printglossary[title=Glossar, style=altlist]
%
%%Abkürzungen ausgeben
%\printglossary[type=\acronymtype,style=long,title=Abkürzungsverzeichnis]

% Anhang
\appendix
\chapter{Anhang}
\section{Ehrenwörtliche Erklärung}
\vspace{2 cm}
Ich erkläre hiermit ehrenwörtlich,
\vspace{1 cm}
\begin{enumerate}
	\item dass ich meine \arbeit ~mit dem Thema
	\begin{center}\glqq\thema\grqq \end{center} ohne fremde Hilfe angefertigt habe,\\
	\item dass ich die Übernahme wörtlicher Zitate aus der Literatur sowie die Verwendung der Gedanken anderer Autoren an den entsprechenden Stellen innerhalb der Arbeit gekennzeichnet habe und\\
	\item dass ich meine \arbeit ~bei keiner anderen Prüfung vorgelegt habe.\\
\end{enumerate}
\vspace{0.5 cm}
Ich bin mir bewusst, dass eine falsche Erklärung rechtliche Folgen haben wird.


\vspace{3 cm}
% Hier kommen die Unterschriten und Ort Datum hin
\begin{tabular}{p{6 cm}p{.5 cm}l}
Gera, \today \\
Ort, Datum 
\end{tabular}
\hfill \hspace{1 cm}
\begin{tabular}{p{7 cm}p{.5 cm}l}
\dotfill \\ 
Unterschrift 
\end{tabular}

\end{document} 
