%Befehle für Glossar
%\newglossaryentry{glos:fab}{
%name=Fabricate,
%description={Beschreibt alles was außerhalb des AIMS\texttrademark \ EUV stattfindet.}
%}
%
%\newglossaryentry{glos:toolpc}{name=Tool-PC, description={Jener Computer, auf dem die AIMS\texttrademark \ EUV Nutzerapplikation läuft.}}
%
%\newglossaryentry{glos:outerhh}{name=OuterHandlingHost, description={Ein Computer, auf dem ein Windows Embedded läuft, der Befehle für die Ansteuerung des OuterHandlings aus dem Netzwerk entgegen nimmt und an die entsprechenden Subsysteme weiter gibt und Statusänderungen des Outer-Handlings den angemeldeten Clients bekannt macht.}}
%
%\newglossaryentry{glos:aimshost}{name=AimsHost, description={Ist ein Dienst, welcher als High-Level-Schnittstelle für die AIMS\texttrademark \ EUV Nutzerapplikation dient, um Maschinenspezifische Befehle weiterzugeben.}}
%
%\newglossaryentry{glos:ll}{name=Load Lock, description={dt.: Ladeschleuse; bezeichnet jene Schleuse, die die Verbindung zwischen Inner-Handling und Outer-Handling herstellt.}}
%
%\newglossaryentry{glos:ad}{name=Anwendungsdomäne, description={Ist eine isolierte Laufzeitumgebung für .NET Anwendungen, um Sicherheit und Zuverlässigkeit für diese Anwendungen zu bieten.}}
%
%\newglossaryentry{glos:framework}{name=Framework, description={Beschreibt ein Programmiergerüst, welches in der Softwaretechnik, insbesondere in der objektorientierten Softwareentwicklung Anwendung findet.}}
%
%\newglossaryentry{glos:remote}{name=Remote, description={Beschreibt alles, was sich außerhalb der Anwendungsdomäne befindet.}}
%
%\newglossaryentry{glos:hap}{name=HAP GmbH Dresden, description={Unternehmen aus Dresden, welches die Ansteuerung des Outer-Handlings mitentwickelt hat.}}
%
%\newglossaryentry{glos:clr}{name=Common Language Runtime, description={Die Laufzeitumgebung von .NET für die Sprachen VisualBasic.NET, C\# und J\#.}}
%
%\newglossaryentry{glos:or}{name=oder, description={Eine Disjuktion (logisches d.h. einschließendes oder). Nicht zu verwechseln mit der Kontravalenz (ausschließendes oder).}}
%
%\newglossaryentry{glos:uri}{name=Uniform Resource Identifier, description={Ist eine Zeichenfolge, die zur eindeutigen Identifizierung einer abstrakten oder physischen Ressource dient. Bsp: https://www.github.com/rails/rails.git}}

%Eine Abkürzung mit Glossareintrag
\newacronym{soa}{SoA}{Struct of Array}
\newacronym{aos}{AoS}{Array of Struct}
